\documentclass{article}
\usepackage[utf8]{inputenc}
\usepackage{setspace}
\usepackage[russian]{babel}  
\usepackage{multicol}
\usepackage{ragged2e}
\usepackage{amsmath}
\usepackage{graphicx}
\usepackage[absolute,overlay]{textpos}
%\usepackage[texcoord, grid,gridcolor=red!10,subgridcolor=green!10,gridunit=pt] {eso-pic}
\usepackage{hyperref}

\begin{document}
\thispagestyle{empty}
\begin{textblock*}{300pt}(175pt, 80pt)
\begin{center} МИНИСТЕРСТВО ОБРАЗОВАНИЯ И НАУКИ РФ \end{center}

\begin{center} Федеральное государственное автономное \\
образовательное учреждение высшего образования \\
<<Национальный исследовательский университет ИТМО>> \end{center}

\begin{center}\textbf {\footnotesize ФАКУЛЬТЕТ ПРОГРАММНОЙ ИНЖЕНЕРИИ И КОМПЬЮТЕРНОЙ ТЕХНИКИ} \end{center} 
\end{textblock*}

\begin{textblock*}{300pt}(175pt, 140pt)
\begin{center}
\vspace{8em}\textbf{РАСЧЁТНО-ГРАФИЧЕСКАЯ РАБОТА № 1} \\
\vspace{1.2em}
по дисциплине \\
<<МАТЕМАТИКА>> \\

\vspace{3em}Вариант №2 \\
\end{center}
\end{textblock*}

\begin{textblock*}{300pt}(250pt, 350pt)
\vspace{17em}\begin{flushright}\textit{Выполнили:} \\
Студенты группы P3118: \\
Баранов Денис Владимирович \\
Кравец Роман Денисович \\
Платонов Никита Олегович \\
Шипунов Илья  Михайлович \\ 
\vspace{1em}
\textit{Преподаватель:} \\
Беспалов Владимир \\ Владимирович \\
\end{flushright}
\end{textblock*}

\begin{textblock*}{300pt}(175pt, 600pt)
\vspace{7em}\begin{center}
Санкт-Петербург \\
2021
\end{center}
\end{textblock*}

\hspace{0em}
\newpage
\begin{textblock*}{300pt}(105pt, 70pt)
\Large\textbf{Текст задания:}
\end{textblock*}
\begin{picture}(0,0) (35,480) \includegraphics[width = 40em, height = 48em]{images/Task1.png} \end{picture} 
\hspace{0em}
\newpage
\begin{picture}(0,0) (35,450) \includegraphics[width = 40em, height = 50em]{images/Task2.png} \end{picture} 
\hspace{0em}
\newpage

\begin{textblock*}{300pt}(100pt, 70pt)
\Large\textbf{Ход работы:} \\
\vspace{0.8em} \hspace{0.4em}
\large\textit{Задание №1:}
\end{textblock*}

\begin{textblock*}{450pt}(75pt, 120pt)
1,2) \href{https://www.desmos.com/calculator/hbzeucujrj}{Ссылка на DESMOS} \\
Зафиксируем  $\rho = 1,  \alpha = 1$, тогда, при изменении параметра , получаем следующие результаты: \\
При $\varepsilon = 0$: график – окружность, \\
При $\varepsilon \in (-1, 1)$: график – эллипс, \\
При $\varepsilon = -1, 1$: график – парабола, \\
При $\varepsilon \in (-\infty, -1) \cup  (1, \infty)$: график – гипербола; \\
\hspace{0em} \\

Где $\alpha$ - угол между фокальной плоскостью и полярной осью, уменьшение и увеличение которого приводит к вращению графика по и против часовой стрелки соответственно. \\

$\varepsilon$ - эксцентриситет, то есть: числовая характеристика, показывающая степень отклонения конического сечения от окружности, в данной формуле - основной параметр, изменение которого определяет вид получившейся кривой.
\end{textblock*}

\begin{textblock*}{450pt}(75pt, 300pt)
3) \large $x' =r \cos{(\varphi - \alpha)}$  и $y' = r \sin{(\varphi - \alpha)}$ \\
\hspace{0em} \\
\large{$r = \frac{\rho}{1 - \varepsilon\cos{(\varphi - \alpha})} \Rightarrow r(1 - \varepsilon\cos{(\varphi - \alpha})) = \rho \Rightarrow r - \varepsilon r \cos{(\varphi - \alpha}) = \rho \Rightarrow r - \varepsilon x = \rho$} \\
\hspace{0em} \\
По теореме Пифагора: $\sqrt{x^2 + y^2} = r \Rightarrow \sqrt{x^2 + y^2} - \varepsilon x = \rho \Rightarrow \sqrt{x^2 + y^2} = \varepsilon x + \rho$ \\
\hspace{0em} \\
Возведём обе части уравнения в квадрат: $x^2 + y^2 = \rho^2 + 2 \rho \varepsilon x + \varepsilon^2 x^2$ \\
\hspace{0em}\\
$x^2 - 2 \rho \varepsilon x - \varepsilon^2 x^2  + y^2 = \rho^2 \Rightarrow \frac{x^2 - 2 \rho \varepsilon x - \varepsilon^2 x^2}{\rho^2} + \frac{y^2}{\rho^2} = 1$ - \normalsize каноническое уравнение эллипса.
\end{textblock*}

\begin{textblock*}{475pt}(75pt, 425pt)
4) \begin{tabular}{| c | c | c | c |}
\hline
\hspace{0em} & Эксцентриситет, км/c & Афелий, млн. км & Перигелий, млн. км \\ \hline
Плутон & 0.244 & 7304.3 & 4435.0  \\ \hline
Сатурн & 0.057 & 1514.5 & 1352.6\\ 
\hline
\end{tabular} \\
\hspace{0em} \\
Перигелий - точка на орбите планеты, астероида или кометы, ближайшая к Солнцу. $\Rightarrow r_{min}$ \\
Воспользуемся формулой: \large{$r = \frac{\rho}{1 - \varepsilon\cos{(\varphi - \alpha})}$} \normalsize{и подберем такое значение $\cos{(\varphi - \alpha})$, чтобы значение r было минимально. $\Rightarrow \cos{(\varphi - \alpha}) = -1$}.\\
\hspace{0em} \\
Из вышеописанного следует: \large $r_{min} = \frac{\rho}{1 + \varepsilon} \Rightarrow \normalsize {\rho = r_{min}(1 + \varepsilon)}$ \\
Тогда: \\
1) $\rho_1 = 4435.0 \cdot (1 + 0.244) = 5517,14$ (млн. км)\\
2) $\rho_2 = 1352.6 \cdot (1 + 0.057) = 77,0982$ (млн. км)
\end{textblock*}
\hspace{0em}
\newpage
\hspace{0em}
\begin{textblock*}{475pt}(75pt, 70pt)
5) Существует четыре вида космических скоростей: \\

1. Первая космическая скорость или Круговая скорость V1 — скорость, которую необходимо придать объекту без двигателя, пренебрегая сопротивлением атмосферы и вращением планеты, чтобы вывести его на круговую орбиту с радиусом, равным радиусу планеты. \\
$V_1 = \sqrt{G \cdot \frac{M}{r + h}}$, где G - гравитационная постоянная, M - масса планеты, r - радиус планеты, h - расстояние от поверхности планеты до объекта. \\

2. Вторая космическая скорость (параболическая скорость, скорость убегания) — наименьшая скорость, которую необходимо придать объекту (например, космическому аппарату), масса которого пренебрежимо мала относительно массы небесного тела (например, планеты), для преодоления гравитационного притяжения этого небесного тела. \\
$V_2 = \sqrt{2} \cdot V_1$ \\

3. Третья космическая скорость — минимально необходимая скорость тела без двигателя, позволяющая преодолеть притяжение Солнца и в результате уйти за пределы Солнечной системы. \\
$V_3 = \sqrt{(\sqrt{2} - 1)^2 V_Z^2 + V_2^2}$, где $V_Z$ - орбитальная скорость.

4. Четвёртая космическая скорость — минимально необходимая скорость тела без двигателя, позволяющая преодолеть притяжение галактики Млечный Путь. \\

С учётом выше описанного и дополнительной формулы для орбитальной скорости: \\
$V_Z = \sqrt{\mu(\frac{2}{r} - \frac{1}{\alpha})}$, где $\mu$ - гравитационный параметр ($\mu = G(M + m)$), $\alpha$ - длина большой полуоси,\\
можно сделать вывод, что на траекторию движения тела в пределах Солнечной системы влияют следующие физические характеристики тела: его собственная масса, кроме того, следующие физические величины также влияют на траекторию: масса планеты, расстояние между объектом и Солнцем, длина большой полуоси.

\end{textblock*}
\end{document}
